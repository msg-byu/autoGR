\documentclass[12pt,prb,reprint]{revtex4-1}
%% \documentclass[12pt,prb,reprint]{elsarticle}
\usepackage{xcolor}
\usepackage{amsmath}
\usepackage{graphicx}
\usepackage[normalem]{ulem}

\begin{document}


\title{Summary of Mueller, Froyen and AFLOW kpoint tests}

%% \author
%% {Wiley S. Morgan}
%% \affiliation
%% {Department of Physics and Astronomy, Brigham Young University, Provo Utah 84602 USA}

%% \author
%% {Gus L. W. Hart}
%% \affiliation
%% %% \address
%% {Department of Physics and Astronomy Brigham Young University, Provo Utah 84602 USA}


\begin{abstract}

\end{abstract}

\maketitle

%% \section{Introduction} \label{Intro}

%% In computational materials science many materials properties are
%% determined by performing integrals over the Brillouin zone. The
%% accuracy of these calculations is directly related to the symmetries
%% and densities of reciprocal space points (k-points) used to perform
%% the integral. Over the years there have been many attempts to
%% streamline the development of the most efficient k-point sampling so
%% that a researcher can perform efficient calculations without having to
%% spend time determinig what the best k-point grid is. The most recent
%% advancement in this effort, contributed by Wisesa et al.\cite{Wisesa},
%% employed informatics to do a brute force search over possible k-point
%% grids.

%% With this new database avaialble the question becomes how much better
%% are these grids than those typically used in research today? To
%% answere that question we have performed convergence tests for nine
%% pure elements (Al, Cu, K, Pd, Re, Ti, V, W, and Y) using the
%% generalized Monkhorst-Pack (GMP) k-points developed by Wisesa,
%% AFLOW\cite{AFLOW} and the Froyen\cite{Froyen} k-point method in
%% VASP. These tests will demonstrate where the methad developed by
%% Wisesa has advantages over other methods currently employed.

\section{Performance of GMP} \label{Performance}
\subsection{Methodology} \label{Method}
To test the performance of the new Generalized Monkhorst-pack
(GMP)grids we performed VASP calculations on 9 different pure elements
(Al, Cu, K, Pd, Re, Ti, V, W, and Y) over a range of cell sizes and
k-point densities. Each calculation was performed using three
different k-point grids; GMP, a Froyen type grid, and a standard
Monkhorst-pack grid (as implemented in AFLOW). The convergence rate of
each k-grid method were then compared to determine which, on average,
had the fastest convergence.

The POSCARs for these calculations were generated by
enumerating\cite{enum3} a binary system of the parent lattice for each
element. For the cubic systems cell sizes from 1 to 11 were generated
for the calculations and for the hexagonal close packed (HCP) systems
cell sizes from 1 to 7 (because HCP has a two atom basis). A candidate
POSCAR was then randomly selected for each cell size and element.

K-point grids from 3 to 33 cubed k-points per reciprocal atom were
used to perform the calculations of the systems with a cubic primitive
cell and grids from 3 to 11 cubed k-points per reciprocal atom were
used for the HCP systems. For both the GMP and AFLOW generated grids
the offset was determined by the software in use to optimize the
grids. In the case of Froyen grids the multiple basis options were
used to constructe the grids; simlpe cubic (SC), base centered cubic
(BCC), and face centered cubic (FCC) for systems with a cubic
primitive cell and HCP grids for systems with a hexagonal primitive
cell. The offsets used for the cubic systems were (0.5, 0.5, 0.5) for
SC and FCC, (0, 0, 0) for BCC and (0, 0, 0.5) for HCP. These offsets
were used in an attempt to optimize the symmetry reduction in each
case.

For each POSCAR selected VASP calculations were performed with k-point
grids of densities 3 to 22 cubed k-points per reciprical atom. For
AFLOW and the GMP k-point grids it was sufficient to specify the
desired number of k-points in the input files and allow the codes to
generate the appropriate k-point grids. For the Froyen method grids
had to be generated by hand for the most common k-point lattices that
were commencerate with the parent lattice. That is for the cubic
parent lattices k-point grids that had simple cubic, face-centered
cubic, and body-centered cubic lattices were used with on offset from
gamma of (0.5, 0.5, 0.5) while for the HCP parent lattice 4 different
HCP k-point grids were generated each having a different offset from
gamma (the offsets for the 4 cases were HCP1: (0.5, 0.5, 0.0), HCP2:
(0.0, 0.0, 0.5), HCP3: (0.5, 0.5, 0.5), and HCP4: (0.0, 0.0, 0.0)).

Each system was allowed only a single run of the VASP code to
relax. For the cubic elements the converged energy value was
considered to be the energy of a single atom cell with 22 cubed
kpoints per reciprocal atom. For the elements that have a HCP parent
lattice the converged energy was considered to be the average energy
found from all four offstets at the highest k-point density for the
Froyen cases. {\color{red}Since the converged energy for the HCP cases is an
average value we have excluded any VASP calculations that converged to
within a standard deviation of the averaged energy from our dataset.}

Potentials were constructed using PAW\_PBE potentials repeated twice
in each POSCAR so that the calculations are actually for pure
elements. 

\subsection{Comparison of Methods} \label{comparison}

All results from the VASP calculations can be found in
Fig.~\ref{fig:AllvGMP}. The figure displays the error in the enery
calculated vs the ratio of irreducible k-points from a method relative
to GMP. For example to get an error of 1E-3 in the energy calculation
a Froyen BCC or FCC grid would need 100 times as many irreducible
k-points as GMP. From this plot we can clearly see that for cubic
systems the GMP vastlp outperforms any other method in use. However,
when we consider the HCP systems we see a drastic drop in
performance. In order to better understand this trend we investigate
the k-points selected by GMP for these systems.

\begin{figure} %% [h]
  \includegraphics{../plots/All_vs_Mueller.pdf}
  \centering
  \caption{A plot of the convergence of the different k-point selction
    methods as compared to GMP. The error in the energy relative to
    the converged value is displayed on the x-axis. The y-axis relates
    the ratio of the number of irreducible k-points needed to reach
    the desired accuracy.}
  \label{fig:AllvGMP}
\end{figure}

\section{Investigation of HCP} \label{HCP}

%% As we saw in Sec.~\ref{comparison} GMP does not perform as well for
%% HCP grids as it does for other systems. In order to unerstand this
%% better we will study the k-points selected by GMP for a 2-atom HCP
%% cell at a number of k-point densities. For ease of analysis we will
%% investigate GMP grids that include the gamma point. For the
%% investigation we will look at k-point densities of 88 and 536 k-points
%% in the integration cell.

%% In the case of 88 k-points GMP provides the following k-points in
%% reciprocal coordinates:

%% \begin{equation}
%%   0.00000000000000 0.00000000000000 0.00000000000000 1.0 ! 1
%%   0.25000000000000 0.62500000000000 0.50000000000000 4.0 ! 2
%%   0.50000000000000 0.25000000000000 0.00000000000000 2.0 ! 3
%%   0.00000000000000 0.50000000000000 0.00000000000000 1.0 ! 4
%% \end{equation}

%% These vectors are all linearly dependent, that is
%% $(0.5,0.25,0)=2*(0.25,0.625,0.5) mod 1$, where the mod 1 maps the new
%% vector back inside the cell. We can then conclude that the vector
%% $(0.25,0.625,0.5)$ is a basis vector for this grid since all other
%% points can be constructed from it. The other two vectors in the basis
%% must simultaneasly be able to map k-points back into the cell and be
%% equivalent to a provided point (otherwise they would point to
%% additional k-points in the irreducible zone). These basis for this
%% grid must there be:

%% \begin{equation}
%%   \label{eq:grid1}
%%   S =((0.25,0.625,0.5),(1,0,0),(0,0,1))
%% \end{equation}

%% Now that we have a basis for this grid we want to ensure that the grid
%% preserves the symmetry of the original parent lattice and that the
%% ratio of reducible k-points to irreducible k-points is less than the
%% total number of symmetry operations for the parent cell. The first
%% check will ensure that the provided grid will have the maximal
%% possible reduction of k-points for the cell. The second will verify
%% that the reduction we see is appropriate for the system.

%% In order to verify the first condition we need to review some basics
%% of group theory and the symmetry group. Each crystal class has a point
%% group $G$ that contains all the symmetries of the crystal. Hexagonal
%% lattices have a point group consiting of 24 different operations. Each
%% of these 24 operations can be recustructed from the generators $Gs$ of
%% the point group by taking all possible combinations of the
%% generators. We state without proof that if a superlattice, $B$, has
%% the same symmetry as it's parent lattice $A$, where $B=AH$ and $H$ is
%% an integer matrix, then:

%% \begin{equation}
%%   \label{eq:sym_pres}
%%   \Z = B^{-1}gB  \forall g \in Gs
%% \end{equation}

%% where $\Z$ is an integer matrix. In order to apply this relation to
%% the basis vectors identified in Eq.~\ref{eq:grid1} we make use of the well
%% known relation between a supercell and a reciprocal grid, $S =
%% (B^{-1})^T$. Applying this check to basis produces the following
%% matrices:

%% \begin{equation}
%%   \Z = \begin{pmatrix} -1 & 0 & 0 \\ 0 & -1  & 0 \\ 0 & 0 & -1 \end{pmatrix}, 
%%   \begin{pmatrix} -1 & 0 & 0 \\ 0 & 1  & 0 \\ 0 & 0 & -1  \end{pmatrix}, 
%%   \begin{pmatrix} 0.8 & 0 & -0.6 \\ 0 & -1  & 0 \\ -0.6 & 0 & -0.8\end{pmatrix}
%% \end{equation}

%% As we can see the third matrix is not an integer matrix. We can
%% conclude that the basis identified in Eq.~\ref{eq:grid1} does not have
%% the same symmetry as the parent.

%% Moving on to the second condition will allow us to determine if the
%% symmetry of the k-point grid is a subset of the original hexagonal
%% lattice. The total number of k-points in this case is 88. If we ignore
%% the gamma point, $(0,0,0)$, then there are only irreducible k-points
%% that contributed to the reduction. The ratio of the total number of
%% k-points and the irreducible number of k-points is then:

%% \begin{equation}
%%   \label{eq:HCP1_rat}
%%   \frac{88,3}=29 \frac{1,3}
%% \end{equation}

%% Apparently every k-point provided by the GMP method has 29
%% symmetrically equivalent points. An impressive feat once we consider
%% that hexagonal lattices have only 24 symmetry operations! In fact the
%% only crystal classes with more symmetry than hexagonal lattices are
%% the cubic systems.

%% To ensure that this isn't a random occurance let us consider the
%% second case of 536 k-points. In this case GMP provides the following
%% k-points:

%% \begin{equation}
%%   0.00000000000000 0.00000000000000 0.00000000000000 1.0 ! 1
%%   0.16666666666667 0.91666666666667 0.50000000000000 2.0 ! 2
%%   0.33333333333333 0.83333333333333 0.00000000000000 2.0 ! 3
%%   0.50000000000000 0.75000000000000 0.50000000000000 2.0 ! 4
%%   0.66666666666667 0.66666666666667 0.00000000000000 2.0 ! 5
%%   0.83333333333333 0.58333333333333 0.50000000000000 2.0 ! 6
%%   0.00000000000000 0.50000000000000 0.00000000000000 1.0 ! 7
%%   0.00000000000000 0.12500000000000 0.50000000000000 2.0 ! 8
%%   0.16666666666667 0.04166666666667 0.00000000000000 4.0 ! 9
%%   0.33333333333333 0.95833333333333 0.50000000000000 4.0 ! 10
%%   0.50000000000000 0.87500000000000 0.00000000000000 4.0 ! 11
%%   0.66666666666667 0.79166666666667 0.50000000000000 4.0 ! 12
%%   0.83333333333333 0.70833333333333 0.00000000000000 4.0 ! 13
%%   0.00000000000000 0.62500000000000 0.50000000000000 2.0 ! 14
%%   0.00000000000000 0.25000000000000 0.00000000000000 2.0 ! 15
%%   0.16666666666667 0.16666666666667 0.50000000000000 4.0 ! 16
%%   0.33333333333333 0.08333333333333 0.00000000000000 4.0 ! 17
%%   0.50000000000000 0.00000000000000 0.50000000000000 2.0 ! 18
%% \end{equation}

%% From these the following three vectors form a basis:

%% \begin{equation}
%%   S = ((0.16666666666667, 0.91666666666667, 0.5),(0.5,0.75,0.5),(0,0.125,0.5))
%% \end{equation}

%% using this basis inside the condition provided by
%% Eq.~\ref{eq:sym_pres} we can check if this denser grid has the same
%% symmetry as the hexagonal parent lattice. The results are:

%% \begin{equation}
%%   \Z = \begin{pmatrix} -1 & 0 & 0 \\ 0 & -1  & 0 \\ 0 & 0 & -1 \end{pmatrix}, 
%%   \begin{pmatrix} -1 & 0 & 0 \\ -0.43 & 1  & -0.29\\ 0 & 0 & -1  \end{pmatrix}, 
%%   \begin{pmatrix} -1 & 0 & 0 \\ 0.43 & -1  & -0.29 \\ 3 & 0 & 1\end{pmatrix}, 
%% \end{equation}

%% As we can see these matrices have non-integer values and we can
%% conclude that this grid also does not have the symmetry of the parent
%% lattice. For the second condition the total number of k-points is 536
%% and the number of irreducible k-points is 17, ignoring the gamma point. 

%% \begin{equation}
%%   \frac{536/17} = 31 \frac{9/17}
%% \end{equation}

%% Once again the average number of k-points that are symmetrically
%% equivalent to the irreducible k-points is larger than the symmetry
%% group.

%% From these two cases it seems apparent that the symmetries being
%% applied to the hexagonal grids are not correct. Based off the number
%% reduction for the two cases of roughly 29 and 31 it would seem that
%% the symmetry group in use is most likely from one of the cubic crystal
%% classes. This discrepancy likely explains why the HCP systems studied
%% here did not perform as well with the GMP grids as the cubic systmse
%% did.

\section{Conclusion} \label{conclusion}

We have seen that in general the generalized Monkhorst-Pack (GMP)
grids outperform other k-point selection methods. For cubic systems
GMP grids can acheive the same accuracy with 10 to 100 times fewer
irreducible k-points that other methods. %% However, for hexagonal
%% systems the improvement is not as impressive acheiving the same
%% accuracy with 2 to 10 times fewer k-points. Our investigation of the
%% GMP grids generated reveals that this is likely due to an error in the
%% symmetry group used for hexagonal lattices resulting in less efficient
%% k-point grids.

\section{Acknowledgements}
This work was supported under: ONR (MURI N00014-13-1-0635).

\section{References}

\bibliographystyle{unsrt}
\bibliography{kp}

\end{document}
