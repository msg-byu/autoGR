% ****** START of file apssamp.tex ******
%
%   This file is part of the APS files in the REVTeX 4.1 distribution.
%   Version 4.1r of REVTeX, August 2010
%
%   Copyright (c) 2009, 2010 The American Physical Society.
%
%   See the REVTeX 4 README file for restrictions and more information.
%
% TeX'ing this file requires that you have AMS-LaTeX 2.0 installed
% as well as the rest of the prerequisites for REVTeX 4.1
%
% See the REVTeX 4 README file
% It also requires running BibTeX. The commands are as follows:
%
%  1)  latex apssamp.tex
%  2)  bibtex apssamp
%  3)  latex apssamp.tex
%  4)  latex apssamp.tex
%

\documentclass[ reprint,
%superscriptaddress,
%groupedaddress,
%unsortedaddress,
%runinaddress,
%frontmatterverbose, 
%preprint,
%showpacs,preprintnumbers,
%nofootinbib,
%nobibnotes,
%bibnotes,
 amsmath,amssymb, aps,
%pra,
prb,
%rmp,
%prstab,
%prstper,
%floatfix,
showpacs ]{revtex4-1}

\usepackage{graphicx}% Include figure files \usepackage{dcolumn}
%Align table columns on decimal point \usepackage{bm}% bold math
\usepackage{color} \usepackage{float} \usepackage{verbatim}
\usepackage{braket} \usepackage[mathscr]{euscript}
\usepackage{silence} \WarningFilter{revtex4-1}{Repair the float}
%\usepackage{hyperref}% add hypertext capabilities
%\usepackage[mathlines]{lineno}% Enable numbering of text and display math
%\linenumbers\relax % Commence numbering lines

%\usepackage[showframe,%Uncomment any one of the following lines to test 
%%scale=0.7, marginratio={1:1, 2:3}, ignoreall,% default settings
%%text={7in,10in},centering,
%%margin=1.5in,
%%total={6.5in,8.75in}, top=1.2in, left=0.9in, includefoot,
%%height=10in,a5paper,hmargin={3cm,0.8in},
%]{geometry}

% Defining new commands
\newcommand{\jer}{\color{blue}}

\begin{document}

\preprint{APS/123-QED}

\title{Improvements to Generalized Monkhorst-Pack Grids}
% Force line breaks with\\
%\thanks{}

\author{Wiley S. Morgan, Jeremy J. Jorgensen, Brett C. Hess, Branton
  J. Campbell, Gus L. W. Hart} \affiliation{Department of Physics and
  Astronomy, Brigham Young University, Provo, Utah, 84602, USA}
%Lines break automatically or can be forced with \\

% \collaboration{\noaffiliation}

\date{\today}% It is always \today, today, % but any date may be
% explicitly specified
\begin{abstract}

  
\end{abstract}

%\pacs{Valid PACS appear here}% PACS, the Physics and Astronomy %
% Classification Scheme.
% \keywords{Suggested keywords}%Use showkeys class option if keyword
%display desired
\maketitle

\section{Introduction}
High throughput materials design has become a effective route to
material discovery with many successes already documented
\cite{greeley2006computational, hanak2004quantum,
  green2013applications, xiang1995combinatorial,
  jandeleit1999combinatorial, senkan1998high}. The creation of large
material databases is central to the success of high throughput
approaches. Computationally expensive electronic structure codes
generate the data repositories that are analyzed in computational
materials design and limit the extent to which data analysis tools,
such as machine learning, can be applied. Any improvements in the
efficiency of these codes have the potential to significantly increase
the size of theoretical material repositories thus increasing the
accuracy of material predictions.

All electronic structure codes perform numerical integrals over the
Brillouin zone, which in the case of metals are exceedingly slowly
convergent. Dense sampling of the Brillouin zone, required for high
accuracy, is computationally expensive, especially when implementing
hybrid functionals or pertubative expansions in density functional
theory (DFT) \cite{berland2017enabling}. The selection of
$\mathbf{k}$-points covering the Brilloun zone in programs running DFT
haven't changed much since Monkhorst and Pack published their
influential paper over 40 years ago \cite{monkhorst1976special}. Their
method was quickly accepted by the community due it its simplicity and
ability to generalize previous methods \cite{baldereschi1973mean,
  chadi1973special}. Methods that made improvements to Monkhorst-Pack
grids are far less prevalent \cite{froyen1989brillouin, moreno1992optimal,
  wisesa2016efficient}. Resolving the issues pertaining to these
improvements is the topic of this paper.

\section{Background}

Over the past 40 years a variety of $\mathbf{k}$-point selection
methods have been investigated in the literature
\cite{baldereschi1973mean, chadi1973special, monkhorst1976special,
  froyen1989brillouin, moreno1992optimal, wisesa2016efficient}. Many
of these so-called special point methods focused on selecting points
that accurately determined the mean value of a function defined over
the Brillouin zone, since the integral of a periodic function over one
period is simply its mean value. Other factors that were considered in
developing special point methods were selecting the most uniform grids
possible, and exploiting symmetry to the fullest extent possible.

Baldereschi introduced the mean-value point of the Brillouin zone
\cite{baldereschi1973mean}, one of the first special point methods. He
began by performing a Fourier expansion of the periodic, Brillouin
zone function
\begin{equation}
f(\mathbf{k}) = \sum_n c_n e^{i \mathbf{k} \cdot \mathbf{R}_n}
\end{equation}
whose integral over one period, the first Brillion zone (BZ), was
given by the common result
\begin{equation}
\int_\text{BZ} f(\mathbf{k}) = \frac{(2\pi)^3}{\Omega} c_0
\end{equation}
where $\Omega$ was the volume of the Brillouin zone and $c_0$ was the
leading coefficient in the Fourier series. He explained that by
choosing points that made as many as possible of the leading terms
evauate to zero, he would obtain an accurate approximation of the
leading Fourier coefficient and the integral he was after.

Chadi and Cohen extended the mean-value point by introducing sets of
points whose weighted sum eliminated the contribution of a greater
number of leading basis functions than that of the mean-value point
\cite{chadi1973special}. Their grid of $\mathbf{k}$-points could also
be made as dense as required for a desired. Chadi and Cohen grids were
equivalent to uniform grids within the Brillouin zone.

The most popular $\mathbf{k}$-point selection method was created by
Monkhorst and Pack \cite{monkhorst1976special}. They established a
grid of points that generalized the mean-value point of Baldereschi
and its extension by Chadi and Cohen, and were equivalent to points
used by Janak et al.\cite{janak1971gilat}.

Froyen generalized Monkhorst-Pack points, which he called Fourier
quadrature points, by eliminating the restriction that the vectors
that defined the grid be parallel to the reciprocal lattice vectors
\cite{froyen1989brillouin}. However, he did require the grid to be
commensurate with the reciprocal lattice and the have full point-group
symmetry of the crystal.

Moreno and Soler introduced the idea of searching for
$\mathbf{k}$-point grids with the fewest points for a given length
cutoff---a parameter that characterized the quality of the grid and
was closely related to the $\mathbf{k}$-point density)
\cite{moreno1992optimal}. These two conditions meant the grid was as
uniform as possible and that the grid subcells were as spherical as
possible. They argued that since there was no knowledge of how the
function behaved beforehand, and hence no reason to sample one region
over another, uniform grids were. Moreno and Soler further improved
Brillouin zone sampling by finding the optimal offset of the origin
that maximized the symmetry reduction of the $\mathbf{k}$-points.

In a recent paper by Wisesa et al., they acknowledge that the lack of
popularity with Moreno and Soler's approach is due to the
expensiveness in calculating many Froyen grids---which they called
generalized Monkhorst-Pack grids---and searching through them for the
one with the highest symmetry reduction
\cite{wisesa2016efficient}. Their remedy involved precalculating the
grids and searching through them for the ones with the highest
symmetry reduction. These grids were stored in a database that had to
be queried with every DFT calculation. They were able to obtain the
grid with the highest symmetry reduction in a fraction of a second
\cite{mueller2016tool}.

In a recent paper by Wisesa et al., a search through a
$\mathbf{k}$-point grid database was used to significantly accelerate
DFT calculations \cite{wisesa2016efficient}. The database contained
grids that were commensurate with all possible types of lattices and
were guaranteed to have the largest possible symmetry reduction. Their
method of generating the grids was similar to that of Froyen as well
as Moreno and Soler.

\bibliography{bib.bib}{} \bibliographystyle{unsrt}

\end{document}
